\section{Синхронізація}
	
	Досліджуваний сигнал є протяжним в часі, через що повне його відображення на малому екрані буде непрактичним. Натомість, осцилограф будує періодичну розгортку, використовуючи сигнал синхронізації. При правильному його налаштуванні, траекторія руху променя по екрану в кожному циклі буде однаковою, що даватиме стабільне зображення одного або декількох періодів.
	
	Для цього має бути налаштована схема синхронізації -- подія, при настанні якої, промінь почне будувати розгортку заново. Tektronix TDS 2024C \cite{oscilloscope} підтримує три схеми синхронізації: по фронту, по тривалості імпульсу та відеосинхронізацію. В даній роботі буде розглянута лише перша. При синхронізації по передньому (задньому) фронту, умовою початку нового періоду є проходження сигналу зі зростанням (спаданням) через певне значення, що задається окремо.
	
	\subsection{Як здійснити синхронізацію?}
    Панель осцилографу має частину  <<\textbf{ЗАПУСК}>>, де розміщені елементи налаштування синхронізації. Перейшовши у меню синхронізації (кнопка <<\textbf{МЕНЮ CИНХ}>>), використовуючи приекранні кнопки встановили наступні параметри:
    \begin{itemize}
        \item \textbf{Trigger type}: Edge trigger
        \item \textbf{Trigger coupling}: AC
        \item \textbf{Source}: канал, до якого підключений генератор
    \end{itemize}
    Також встановили значення <<\textbf{Trigger level}>> за допомогою ручки <<\textbf{УРОВЕНЬ}>> на 0 В.