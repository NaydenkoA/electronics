\likechapter{Вступ}

Чотирополюсники - елементи, що мають 4 ніжки, призначені для видозмінення вхідного сигналу відповідно до потреб автора. Тому надзвичайно корисно погратися з такою милою іграшкою як при моделюванні на комп'ютері, що дасть змогу дослідити різні режими його роботи, недоступні для прямих вимірів, так і спробувати скласти необхідну схему та зняти відповідні дані своїми багатофункціональними клєшнями.

У даній роботі було експериментально досліджено властивості та поведінку чотирьохполюсника, що скаладається аж з двох елементів позаземної природи: резистор та конденсатор. Відповідно було доступно 2 конфігурації його роботи: інтегруюча та диференціююча.

\textbf{Об'єктом} проведеного дослідження є бідолага чотирьохполюсник, якому у якості заспокоючого припаяли резистор номіналом $200 k\Omega$ та конденсатор $154 nF$. 

\textbf{Предметом} роботи є дослідження вихідного сигналу в залежності від параметрів вхідного. Фазо-частотна та амплітудно частотна характеристика.

\textbf{Метою} роботи ознайомлення жалких студентиків з чудом новітнього світу, провідним зразком надсучасної електроніки, неповторним чотирьохполюсником.

Поставлено наступні \textbf{задачі}: 
\begin{enumerate}
\item Осягнути надскладну методитку підключення такої штуки
\item Подати на неї квадратний сигнал та наглядно зрозуміти, які перетворення відбуваються з ним
\item Подати синусоїдальний сигнал різної частоти на вхід та зняти перетворений сигнал на виході.
\item Навчитися творити дивовижні речі у неймовірній програмі для моделювання електричних схем qucs \cite{qucs}
\item Повторити наш неосяжний результат у програмі для моделювання та порівняти отримані результати.
\end{enumerate}
