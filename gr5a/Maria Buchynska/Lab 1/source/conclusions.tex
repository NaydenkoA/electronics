\likechapter{Висновки}

В ході виконання даної лабораторної роботи ми навчилися користуватись осцилографом, тим самим піднявши настрій одному із авторів на цілий тиждень. Аппарат вмикався в одноканальному та двоканальному режимах роботи. Перший дозволив нам зняти спектри гармонік, що формували подані на осцилограф сигнали. Другий режим дав можливість побудувати фігури Ліссажу, форма яких залежить від співвідношення частот. Потрібно відзначити мужність даного приладу, який ні разу навіть не натякнув про те, що з нього занадто сильно знущаються, звісно, не прочитавши жодної інструкції чи даташиту.

Під час виконання другої частини роботи, а саме, використанні імпедансметра для вивчення активного і реактивного опору резистора, конденсатора і котушки, ми дослідили імпеданс та ємність (індуктивність) для резистора, конденсатора та котушки. Було пояснено природу спостережуваних явищ та отриманих \sout{манускриптів} графіків.

Таким чином, ми освоїли нові прилади і змогли дати пояснення цікавим явищам, про які нам не розповідали ані у школі, ані в університеті.

