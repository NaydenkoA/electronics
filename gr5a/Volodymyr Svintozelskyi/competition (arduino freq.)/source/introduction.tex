\likechapter{Вступ}

Отримання сигналів, частота яких співмірна із частотою роботи тактового генератора, підключенного до мікроконтроллера Atmega, може бути досить цікавою задачею. Фізичний максимум - $\omega = \omega_{gen}/2$ пов'язаний із необхідністю почергового притягання ножки до високого та низького потенціалів, на кожний з таких переходів витрачається деякий мінімальний час $\tau \approx 1/\omega_{gen}$. 

Розумний метод досягнення такої цілі - використання таймерів, що базуються на перериваннях мікроконтроллера atmega328, який використовується на платах arduino uno. Перевагою такого методу є те, що контроллер може спокійно виконувати інші завдання, а в певні моменти часу, визначені таймером, переривати виконання основної програми для генерування нашого сигналу.