\likechapter{Висновки}

В даній лабораторній роботі було досліджено поведінку чотирьохполюсників двох типів: інтегрувальний та диферинціювальний. Отримано експериментальним шляхом їхні АЧХ та ФЧХ. За допомогою останніх було показано вплив паразитних опорів та ємностей на роботу чотирьохполюсника. 

Було усвідомлено надзвичайну користь таких схем: для зглажування сигналу при живленні чутливих контролерів, та, навпаки, для виділення імпульсів та невеликих змін в сигналі.

Окремим плюсом виконання даної лабораторної роботи слід виділити необхідність написання окремої програми для ефективного аналізу даних та побудов необхідних графіків, під час розробки яких студенти сильно покращують свої навички стукати клешнями по клавіатурі їхнього ПК.

Автори набули дуже крутих навичок роботи у програмі моделювання Qucs, \sout{тож тепер вони можуть зібрати власний ПК із г***а та палок, та показати пану Гейтсу, як потрібно розробляти не глючну та неперегружену недолугими функціями та іншими прикростями систему. }