\likechapter{Висновки}
В ході виконання даної лабораторної роботи ми навчилися користуватись осцилографом. Аппарат вмикався в одноканальному та двоканальному режимах роботи. Перший дозволив нам зняти спектри гармонік, що формували подані на осцилограф сигнали. Другий режим дав можливість побудувати фігури Ліссажу, форма яких залежить від співвідношення частот.

Під час виконання другої частини роботи, а саме, використанні імпедансметра для вивчення активного і реактивного опору резистора, конденсатора і котушки, ми отримали здебільшого очікувані результати. Резистор поводив себе як резистор майже у вcьому спектрі частот. Конденсатор теж не викликав багато питань, а ось котушка мала на певній частоті досить неочікуваний пік, а її імпеданс з індуктивного ставав ємнісним. Пояснення цього явища виявилось досить простим: оскільки жодний фізичний прилад не можна вважати ідеальним, то котушка проявляє себе на певній частоті (її називають власною частотою даної котушки індуктивності) як схема RLC елементів. Прийнято вважати, що ємність виникає між витками котушки і створює спостережуваний ефект, а опір пояснює певні втрати на даному елементі.  

Таким чином, ми освоїли нові прилади і змогли дати пояснення цікавим явищам, про які нам не розповідали ані у школі, ані в університеті.

